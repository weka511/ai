\documentclass[]{article}
\usepackage{chronology}
\usepackage{float}
\usepackage{caption}
\usepackage{subcaption}
\usepackage{graphicx}
\usepackage{url}
\usepackage{amsmath}
\usepackage{amssymb}
\usepackage{amsthm}
\usepackage{tocloft}
\usepackage{cancel}
\usepackage{thmtools}
\usepackage[toc,nonumberlist]{glossaries}
\usepackage{glossaries-extra}
\newcommand\numberthis{\addtocounter{equation}{1}\tag{\theequation}}
\newtheorem{defn}{Definition}
\newtheorem{thm}{Theorem}
\newtheorem{lemma}[thm]{Lemma}
\graphicspath{{figs/}}
\widowpenalty10000
\clubpenalty10000

\makeglossaries
%opening
\title{Active inference}
\author{Simon Crase}

\begin{document}
	
\newglossaryentry{gls:sde}{
	name={stochastic differential equation},
	description={
	A stochastic differential equation is a differential equation whose coefficients are random numbers or random functions of the independent variable (or variables)}}

\newglossaryentry{gls:helmholz}{
	name={Helmholtz Decomposition},
	description={The Helmholtz Decomposition Theorem, or the fundamental theorem of vector calculus, states that any well-behaved vector field can be decomposed into the sum of a longitudinal (diverging, non-curling, irrotational) vector field and a transverse (solenoidal, curling, rotational, non-diverging) vector field\cite{baird2012helmholtz}}}
\maketitle

\begin{abstract}
This document collects my derivations of equations from \cite{friston_life_2013}.
\end{abstract}

\tableofcontents
\listoftheorems
\section{Fokker-Planck}

\begin{thm}[Fokker-Planck equation]
	If $\vec{x}$ satisfies the \gls{gls:sde}
	\begin{align*}
		\dot{\vec{x}} =& f(\vec{x}) + \omega \numberthis \label{eq:dynamics}
	\end{align*}
	then the probability distribution over states is given by:
	\begin{align*}
		\dot{p}(\vec{x}\vert m) =& \nabla \cdot \Gamma \nabla p -\nabla \cdot(f p) \numberthis \label{eq:fokker:planck}
	\end{align*}
	See \cite[(2.2)]{friston_life_2013} and  \cite[Chapter 33]{cvitanovic2005chaos}.
\end{thm}

\begin{proof}
	TBP
\end{proof}

\cite{friston_life_2013,friston2012free}

\begin{thm}[Equilibrium distribution for Fokker-Planck]
	The equilibrium distribution for \eqref{eq:fokker:planck} is given by:
	\begin{align*}
		p(\vec{x}\vert m) =& \exp \big(-G(\vec{x})\big) \text{, where $G$ satisfies}\\
		f =& -(\Gamma + R) \cdot \nabla G
	\end{align*}
	
\end{thm}
\begin{proof}
The equilibrium distribution of \eqref{eq:fokker:planck} satisfies:
	\begin{align*}
		\dot{p}(\vec{x}\vert m) =&0 \text{, i.e.}\\
		\nabla \cdot (\Gamma \nabla p -f p)=&0 \numberthis \label{eq:fokker:plank:eq}
	\end{align*}
	Since $	p(\vec{x}\vert m) \ge 0$, we can define $G()$ such that:
	\begin{align*}
		p(\vec{x}\vert m) =& \exp \big(-G(\vec{x})\big) \text{. Then}\\
		\nabla p =& - \exp \big(-G(\vec{x})\big) \nabla G(\vec{x})\\
		=& - p \nabla G(\vec{x}) \text{, whence \eqref{eq:fokker:plank:eq} becomes}\\
		-\nabla \cdot [(\Gamma \nabla G +f)p]=&0\\
		 [(\Gamma \nabla G +f)p] =& \nabla \times A
	\end{align*}
	Following \cite[(2.3)]{friston_life_2013} and \gls{gls:helmholz}, define:
	\begin{align*}
		f =& -(\Gamma + R) \cdot \nabla G\\
		\Gamma \nabla p - f p =& \Gamma p \nabla G + (\Gamma + R) \cdot \nabla G p\\
		=& R \cdot \nabla G p
	\end{align*}
\end{proof}


%\begin{figure}[H]
%	\caption{Timeline}
%	\begin{subfigure}[b]{\textwidth}
%		\caption{FEP}
%		\begin{chronology}[3]{2006}{2022}{16cm}[\textwidth]
%			\event{2006}{\color{blue}\cite{friston_free_2006}A Free Energy Principle for the brain}
%			\event{2009}{\cite{friston_free-energy_2009}The free-energy principle: a rough guide to the brain}
%			\event{2013}{\cite{friston_life_2013}Life as we Know it}
%			\event{2021}{\cite{friston_interesting_2021}Some Interesting Observations on the Free Energy Principle}
%			\event{2022}{\cite{friston_free_nodate}The free energy principle made simpler but not too simple}
%		\end{chronology}
%	\end{subfigure}
%	\begin{subfigure}[b]{\textwidth}
%		\caption{Fokker-Plank}
%		\begin{chronology}[3]{2006}{2022}{16cm}[\textwidth]
%			\event{2013}{\cite{friston_life_2013}Life as we Know it}
%			\event{2022}{\cite{friston_free_nodate}The free energy principle made simpler but not too simple}
%		\end{chronology}
%	\end{subfigure}
%\end{figure}

\appendix

\printglossaries

% bibliography goes here

\bibliographystyle{unsrt}
\addcontentsline{toc}{section}{Bibliography}
\bibliography{ActiveInference}

\end{document}
