% This document collects my derivations of equations relevant to Active Inference

% Copyright (c) 2022-2024 Simon Crase

% This script is free software: you can redistribute it and/or modify
% it under the terms of the GNU General Public License as published by
% the Free Software Foundation, either version 3 of the License, or
% (at your option) any later version.

% This script is distributed in the hope that it will be useful,
% but WITHOUT ANY WARRANTY; without even the implied warranty of
% MERCHANTABILITY or FITNESS FOR A PARTICULAR PURPOSE.  See the
% GNU General Public License for more details.

% You should have received a copy of the GNU General Public License
% along with this script.  If not, see <https://www.gnu.org/licenses/>.


\documentclass[]{article}
\usepackage{amsmath}
\usepackage{amssymb}
\usepackage{amsthm}
\usepackage[toc,page]{appendix}
\usepackage{cancel}
\usepackage{caption}
\usepackage{chronology}
\usepackage{float}
\usepackage[toc,nonumberlist]{glossaries}
\usepackage{glossaries-extra}
\usepackage{graphicx}
\usepackage{mathrsfs}
\usepackage{mathtools}
\usepackage{subcaption}
\usepackage{thmtools}
\usepackage{tocloft}
\usepackage{url}

\newcommand{\defeq}{\vcentcolon=}
\newcommand\numberthis{\addtocounter{equation}{1}\tag{\theequation}}
\newcommand{\stcomp}[1]{\widetilde{#1}}
\DeclarePairedDelimiterX{\infdivx}[2]{(}{)}{%
	#1\;\delimsize\|\;#2%
}
\newcommand{\infdiv}{D_{KL}\infdivx}
\DeclarePairedDelimiter{\norm}{\lVert}{\rVert}
\newtheorem{defn}{Definition}
\newtheorem{thm}{Theorem}
\newtheorem{lemma}[thm]{Lemma}
\newtheorem{cor}[thm]{Corollary}
\graphicspath{{figs/}}

\widowpenalty=10000
\clubpenalty=10000
\tolerance=1
\emergencystretch=\maxdimen
\hyphenpenalty=10000
\hbadness=10000

\makeglossaries
%opening
\title{Active inference}
\author{Simon Crase}

\begin{document}

\newabbreviation{gls:kl}{KL}{Kullbak-Leibler}
\newabbreviation{gls:ae}{a.e.}{Almost Everywhere}
	
\newglossaryentry{gls:sde}{
	name={Stochastic Differential Equation},
	description={
	A differential equation whose coefficients are random numbers or random functions of the independent variable (or variables)\cite{vankampen2007stochastic}}}

\newglossaryentry{gls:helmholz}{
	name={Helmholtz Decomposition},
	description={Any well-behaved vector field can be decomposed into the sum of a longitudinal (diverging, non-curling, irrotational) vector field and a transverse (solenoidal, curling, rotational, non-diverging) vector field\cite{baird2012helmholtz}}}

\newglossaryentry{gls:kl:divergence}{
	name = {Kullback-Leibler Divergence},
	description={A measure of the difference between two probability distributions
	\begin{align*}
		\infdiv{Q(x)}{P(x)}\defeq&\mathbb{E}_{Q(x)}[Q(x)-P(x)] \numberthis \label{eq:kl:divergence}
	\end{align*}}
}

\maketitle

\begin{abstract}
This document collects my derivations of equations from Karl Friston's work, in order to help me understand the details of the Free Energy Principle.
\end{abstract}

\tableofcontents
\listoftheorems
\listoffigures
\listoftables

\section{Introduction}

Table \ref{table:notation} shows changes to notation during the period shown in Figure \ref{fig:timeline}--the timeline of \cite{beal2003variational,friston_free_2006,friston_free-energy_2009,friston_life_2013,friston2016therefore,blei2017,friston_interesting_2021,friston2022free,parr2022Active}. I have adopted the notation of \cite{parr2022Active}, since I expect it to be stable.

\begin{figure}[H]
	\caption{Timeline for Active Inference references}\label{fig:timeline}
	\begin{chronology}[5]{2002}{2024}{16cm}
		\event{2003}{\cite{beal2003variational}Variational algorithms for approximate Bayesian inference}
		\event{2006}{\color{blue}\cite{friston_free_2006}A Free Energy Principle for the brain}
		\event{2009}{\color{blue}\cite{friston_free-energy_2009}The free-energy principle: a rough guide to the brain}
		\event{2013}{\color{blue}\cite{friston_life_2013}Life as we Know it}
		\event{2016}{\color{blue}\cite{friston2016therefore}I am therefore I think}
		\event{2017}{\cite{blei2017}Variational inference: A review for statisticians}
		\event{2021}{\color{blue}\cite{friston_interesting_2021}Some Interesting Observations on the Free Energy Principle}
		\event{\decimaldate{1}{1}{2022}}{\color{blue}\cite{friston2022free}The free energy principle made simpler but not too simple}
		\event{\decimaldate{30}{10}{2022}}{\color{blue}\cite{parr2022Active}Active Inference: The Free Energy Principle in Mind, Brain and Behavior}
	\end{chronology}
\end{figure}

\begin{table}[H]
	\begin{center}
		\caption{Notation for Active Inference references}\label{table:notation}
		\begin{tabular}{|l|c|c|c|c|c|}\hline
			&\cite{beal2003variational}&\cite{friston_life_2013}&\cite{friston2016therefore}&\cite{blei2017}&\cite{friston2022free,parr2022Active}\\ \hline
			External State&x&$\Psi$&$\Psi$&z&$\eta$\\ \hline
			Internal State&&$\lambda$&r&&$\mu$\\ \hline
			Sensory State&y&s&s&x&s\\ \hline
			Action State&&a&a&&a\\ \hline
			Autonomous States&&&&& $\alpha(a,\mu)$\\\hline
			Blanket States&&&&&$b(s,a)$\\ \hline
			Particular States&&&&&$\pi(b,\mu)$\\ \hline
			Ergodic Density&p&$p(\psi,s,a,\lambda\vert m)$&$p(\psi,s,a,r\vert m)$&p&p\\ \hline
			Recognition or Variational Density&q&$q(\psi\vert \mu)$&$q(\psi\vert \mu)$&q&q\\ \hline
		\end{tabular}
	\end{center}
\end{table}



\section{Free Energy}
This section makes use of the following Theorem\cite{parr2022Active}.

\begin{thm}[Bayes' Rule]
	\begin{align*}
		 P(x)P(y\vert x) = P(x,y) = P(y)P(x\vert y) \numberthis \label{eq:bayes}
	\end{align*}
\end{thm}

\subsection{Variational Free Energy}

We derive the equation of \cite[Section 2.5]{parr2022Active}, which defines Variational Free Energy. 
\begin{align*}
	F[Q,y] \defeq&\underbrace{ -\mathbb{E}_{Q(x)}[\ln{P(y,x)}] }_\text{Energy}- \underbrace{\mathbb{H}[Q(x)]}_\text{Entropy} \text{, where} \numberthis \label{eq:fe}\\
	\mathbb{H}[Q(x)] \defeq& - \mathbb{E}_{Q(x)} [\ln{Q(x)}] \text{, the Entropy defined in \cite{shannon1949mathematical}}
\end{align*}
\begin{enumerate}
	\item Energy--$P \propto e^{-E}$
	\item Entropy
\end{enumerate}
\begin{align*}
	F[Q,y] = & -\mathbb{E}_{Q(x)}[\ln{P(y,x)}] + \mathbb{E}_{Q(x)}[\ln{Q(x)}]\numberthis \label{eq:fe:int}\\
	=&-\mathbb{E}_{Q(x)}[\ln{P(y\vert x)} + \ln{P(x)}] + \mathbb{E}_{Q(x)}[\ln{Q(x)}] \text{, from \eqref{eq:bayes}} \\
	=&\mathbb{E}_{Q(x)}[\ln{Q(x)}-\ln{P(x)}] -\mathbb{E}_{Q(x)}[\ln{P(y\vert x)}]\\
	=&\underbrace{ D_{KL}[Q(x)\lvert\rvert P(x)]}_\text{Complexity} - \underbrace{\mathbb{E}_{Q(x)}[\ln{P(y\vert x)}]}_\text{Accuracy}\text{, using \eqref{eq:kl:divergence}} \numberthis \label{eq:fe:complexity:accuracy}
\end{align*}
\begin{enumerate}
	\item Complexity--The \gls{gls:kl:divergence} measures the difference between P and Q
	\item Accuracy
\end{enumerate}
Rearranging \eqref{eq:fe:int} and applying  \eqref{eq:bayes}:
\begin{align*}
	F[Q,y] =& -\mathbb{E}_{Q(x)}[\ln{P(x\vert y)} + \ln{P(y)}] + \mathbb{E}_{Q(x)}[\ln{Q(x)}]\\
	=& \mathbb{E}_{Q(x)}[\ln{Q(x)}-\ln{P(x\vert y)]} - \mathbb{E}_{Q(x)}\ln{P(y)}\\
	=&\underbrace{D_{KL}[Q(x)\lvert\rvert P(x\vert y)]}_\text{Divergence} - \underbrace{\ln{P(y)}}_\text{Evidence} \numberthis \label{eq:fe:divergence:evidence}\\
	\ge& -\ln{P(y)} \text{, from Theorem \ref{thm:kl:nn}} \numberthis \label{eq:evidence:bound}
\end{align*}

\begin{enumerate}
	\item Divergence
	\item Evidence
\end{enumerate}

\begin{thm}[\gls{gls:kl} is non-negative]\label{thm:kl:nn}
	\begin{enumerate}
		\leavevmode
		\item 	$\infdiv{Q(x)}{P(x)}\ge 0$
		\item 	$\infdiv{Q(x)}{P(x)}= 0 \iff Q(x)=P(x)$ \gls{gls:ae}
	\end{enumerate}
\end{thm}
\subsection{Expected Free Energy}
This Section derives the equations of \cite[Section 2.6]{parr2022Active}. We use $\tilde{x}$ to denote a sequence, and $\pi$ a policy.

Expected Free Energy is defined in \cite[Section 2.6]{parr2022Active} as:
\begin{align*}
	G(\pi) \defeq& -\underbrace{\mathbb{E}_{Q(\tilde{x},\tilde{y},\vert \pi)}\big[\infdiv{Q(\tilde{x},\tilde{y},\vert \pi)}{Q(\tilde{x}.\vert \pi)}\big]}_\text{Information gain}  - \underbrace{\mathbb{E}_{Q(\tilde{y},\vert \pi)}\big[\ln{P(\tilde{y}\vert C)}\big]}_\text{Pragmatic value}\\
	Q(\tilde{x},\tilde{y},\vert \pi) \defeq& Q(\tilde{x}\vert \pi)P(\tilde{y}\vert\tilde{x})
\end{align*}

\begin{appendices}
	
	\section{Fokker-Planck Equation}
	
	\begin{thm}[Fokker-Planck equation]
		If $\vec{x}$ satisfies the \gls{gls:sde}
		\begin{align*}
			\dot{\vec{x}} =& f(\vec{x}) + \omega \numberthis \label{eq:dynamics}
		\end{align*}
		then the probability distribution over states is given by:
		\begin{align*}
			\dot{p}(\vec{x}\vert m) =& \nabla \cdot \Gamma \nabla p -\nabla \cdot(f p) \numberthis \label{eq:fokker:planck}
		\end{align*}
		See \cite[(2.2)]{friston_life_2013} and  \cite[Chapter 33]{cvitanovic2005chaos}.
	\end{thm}
	
	\begin{proof}
		\cite[Section 33.3]{cvitanovic2005chaos}
	\end{proof}
	
	\cite{friston_life_2013,friston2012free}
	
	\begin{thm}[Equilibrium distribution for Fokker-Planck]
		The equilibrium distribution for \eqref{eq:fokker:planck} is given by:
		\begin{align*}
			p(\vec{x}\vert m) =& \exp \big(-G(\vec{x})\big) \text{, where $G$ satisfies}\\
			f =& -(\Gamma + R) \cdot \nabla G \text{, where} \numberthis \label{eq:f:Gamma:R}\\
			\nabla.R =& 0 \text{ and R antisymmetric}
		\end{align*}
		
	\end{thm}
	\begin{proof}
		The equilibrium distribution of \eqref{eq:fokker:planck} satisfies:
		\begin{align*}
			\dot{p}(\vec{x}\vert m) =&0 \text{, i.e.}\\
			\nabla \cdot (\Gamma \nabla p -f p)=&0 \numberthis \label{eq:fokker:plank:eq}
		\end{align*}
		Since $	p(\vec{x}\vert m) > 0$, we can define:
		\begin{align*}
			G(x) =& - \ln{}p(\vec{x}\vert m) \text{, so}\\
			p(\vec{x}\vert m) =& \exp \big(-G(\vec{x})\big) \text{. Then}\\
			\nabla p =& - \exp \big(-G(\vec{x})\big) \nabla G(\vec{x}) \numberthis \label{eq:grad:p}\\
			=& - p \nabla G(\vec{x})
		\end{align*}
		whence \eqref{eq:fokker:plank:eq} becomes
		\begin{align*}
			-\nabla \cdot [(\Gamma \nabla G +f)p]=&0\\
			(\Gamma \nabla G +f)p =& \nabla \times \vec{A} \text{, for some $\vec{A}$}\\
			=& -p \vec{R} \cdot \nabla G \text{, for some $R$, whence}\\
			f =& -(\Gamma + R) \cdot \nabla G \text{, the \gls{gls:helmholz}-- \cite[(2.3)]{friston_life_2013}}
		\end{align*}
	\end{proof}
	\begin{proof}[Alternative proof using indices.]
		Equation \eqref{eq:f:Gamma:R} expands as 
		\begin{align*}
			f_i =& - \sum_{j}\big(\Gamma_{ij}+R_{ij}\big)\partial_j G \text{, whence}\\
			\big(\Gamma\nabla p - \vec{f} p\big)_i =& \sum_{j}\big[\Gamma_{ij}\partial_j p\big] -  f_i p\\
			=& \sum_{j}\big[\Gamma_{ij}\partial_j p\big] + p\sum_{j}\big(\Gamma_{ij}+R_{ij}\big)\partial_j G  \\
			=& - \cancel{\sum_{j}\big[\Gamma_{ij}p\partial_j G\big]} + p\sum_{j}\big(\cancel{\Gamma_{ij}}+R_{ij}\big)\partial_j G \\
			=& p\sum_{j}R_{ij}\partial_j G
		\end{align*}
		Calculating the divergence:
		\begin{align*}
			\nabla \cdot \big(\Gamma\nabla p - \vec{f} p\big) =& \sum_{i,j} \partial_i \big[p R_{ij}\partial_j G\big]\\
			=& \sum_{i,j} \big[(\partial_i p) R_{ij}\partial_j G + p (\partial_i  R_{ij})\partial_j G +\underbrace{ p R_{ij}(\partial_i \partial_j G)}_{=0}\big]\\
			=& \sum_{i,j} \big[-\underbrace{p (\partial_i G) R_{ij}\partial_j G}_{=0} + p (\partial_i  R_{ij})\partial_j G \big] \\
			=& p \sum_{i,j} \underbrace{(\partial_i  R_{ij})}_{\nabla\cdot R=0}\partial_j G\\
			=& 0
		\end{align*}
		
	\end{proof}
From \cite{friston2016therefore}
\begin{align*}
	F(s,a,r) =& E_q\big[L(x)\big]-H\big[q(\Psi\vert r)\big] \text{, where}\\
	L(x) =& -\ln{p(\psi,s,a,r)}\\
	F(s,a,r) =&  E_q\big[ -\ln{p(\psi,s,a,r)} - \ln{q(\Psi\vert r)}\big]
\end{align*}

From \cite{friston2022free}

\begin{align*}
	F\big(\pi(\tau)\big) =& E_q\big[\ln q\big(\eta(\tau)\big) - \ln p\big(\eta(\tau)\big) - \ln p\big(\pi(\tau)\vert\eta(\tau)\big)\big] \numberthis \label{eq:free:energy}
\end{align*}

We can rearrange \eqref{eq:free:energy} in several different ways.
\begin{align*}
	F\big(\pi(\tau)\big) =& E_q\big[\ln q\big(\eta(\tau)\big) - \ln p\big(\eta(\tau)\big) - \ln p\big(\pi(\tau)\vert\eta(\tau)\big)\big]\\
	=& E_q\big[\ln q\big(\eta(\tau)\big)\big] - E_q \bigg[\ln \big[p\big(\eta(\tau)\big)  p\big(\pi(\tau)\vert\eta(\tau)\big)\big]\bigg]\\
	=& E_q\big[\ln q\big(\eta(\tau)\big)\big] - E_q \bigg[\ln \big[  p\big(\pi(\tau),\eta(\tau)\big)\big]\bigg]\\
	=&-\underbrace{H(q\big(\eta(\tau)\big))}_\text{Entropy}  \underbrace{- E_q \bigg[\ln \big[  p\big(\pi(\tau),\eta(\tau)\big)\big]\bigg]}_\text{Identified with Expected Energy in \cite{friston2022free}}
\end{align*}
This agrees with the thermodynamic definition of Helmholtz Free Energy $U-TS$, where $U$ is the enthalpy and $S$ the entropy, if we set the temperature $T=1$.
\begin{align*}
	F\big(\pi(\tau)\big) =& E_q\big[\ln q\big(\eta(\tau)\big) - \ln p\big(\eta(\tau)\big) - \ln p\big(\pi(\tau)\vert\eta(\tau)\big)\big]\\
	=& E_q\big[\ln \frac{q\big(\eta(\tau)\big)}{p\big(\eta(\tau)\big)}   \big] - E_q \bigg[\ln \big[  p\big(\pi(\tau)\vert\eta(\tau)\big)\big]\bigg]\\
	=& D_{KL}\big(q\big(\eta(\tau)\big)\vert\vert p\big(\eta(\tau)\big)\big)- E_q \bigg[\ln \big[  p\big(\pi(\tau)\vert\eta(\tau)\big)\big]\bigg]\\
	\ge&- E_q \bigg[\ln \big[  p\big(\pi(\tau)\vert\eta(\tau)\big)\big]\bigg]
\end{align*}

\begin{align*}
	F\big(\pi(\tau)\big) =& E_q\big[\ln q\big(\eta(\tau)\big) - \ln p\big(\eta(\tau)\big) - \ln p\big(\pi(\tau)\vert\eta(\tau)\big)\big]\\
	=& E_q\big[\ln q\big(\eta(\tau)\big)  - \ln p\big(\pi(\tau)\vert\eta(\tau)\big)\big] +  E_q\big[- \ln p\big(\eta(\tau)\big)\big]\\
	=& D_{KL}\big(q\big(\eta(\tau)\big)\vert\vert p\big(\pi(\tau)\vert\eta(\tau)\big)\big)+  E_q\big[- \ln p\big(\eta(\tau)\big)\big]
\end{align*}	
	\section{Boolean Algebra}
	We define the symmetric difference of two sets:
	\begin{align*}
		A \triangle B \coloneqq& \big(A \setminus B\big) \bigcup \big(B \setminus  A\big)\\
		=& \big(A \cap \stcomp{B}\big) \bigcup \big(B \cap \stcomp{A}\big) \numberthis \label{eq:symm:diff_regular:diff}\\
	\end{align*}
	We also have two useful distributive laws.
	\begin{align*}
		A \bigcup \big(B \cap C \big) =& \big(A \cup B\big) \bigcap  \big(A \cup C\big) \numberthis \label{eq:dist:cup:cap}\\
		A \bigcap \big(B \cup C \big) =& \big(A \cap B\big) \bigcup  \big(A \cap C\big)	\numberthis \label{eq:dist:cap:cup}	
	\end{align*}
	Applying \eqref{eq:dist:cup:cap} to \eqref{eq:symm:diff_regular:diff}:
	\begin{align*}
		A \triangle B =&\big[\big(A \cap \stcomp{B}\big) \cup B\big] \bigcap \big[ \big(A \cap \stcomp{B}\big) \cup \stcomp A \big] \numberthis \label{eq:symm:diff:expanded}
	\end{align*}
	Splitting the two terms in \eqref{eq:symm:diff:expanded}, applying \eqref{eq:dist:cup:cap}, and canceling intersections with the universal set:
	\begin{align*}
		\big[\big(A \cap \stcomp{B}\big) \cup B\big] =& \big(A \cup B\big) \cancel{\bigcap \big(B \cup \stcomp{B}}\big)\\
		\big[ \big(A \cap \stcomp{B}\big) \cup \stcomp A \big] =& \cancel{\big(A\cup\stcomp{A} \big) \bigcap} \big(\stcomp{B} \cup\stcomp{A}\big) \text{, so \eqref{eq:symm:diff:expanded} becomes}\\
		A \triangle B =& \big(A \cup B\big) \bigcap \big(\stcomp{B} \cup\stcomp{A}\big) \numberthis \label{eq:symm:diff:without:diff}
	\end{align*}

	\begin{thm}[Semi-group property for $\triangle$]
		Sets form a semi-group with identity under $\triangle$.
	\end{thm}
	\begin{proof}
		We need to show that:
		\begin{align*}
			A \triangle B =& B \triangle A \numberthis \label{eq:symm:diff:commutative}\\
			\big(A \triangle B\big) \triangle C =& A \triangle \big(B \triangle C\big) \numberthis \label{eq:symm:diff:associative}
		\end{align*}
		Equation \eqref{eq:symm:diff:commutative} follows from the symmetry of \eqref{eq:symm:diff_regular:diff}.
		
		We split $A,\,B,\, and C$ into non-overlapping parts:
		\begin{align*}
			A =& S_A \cup S_{AB} \cup S_{AC} \cup S_{ABC} \\
			B =& S_B \cup S_{BC} \cup S_{BA} \cup S_{BCA}\\
			C =& S_C \cup S_{CA} \cup S_{CB} \cup S_{CAB}
		\end{align*}
		where
		\begin{align*}
			S_A =& A - B - C\\
			S_{AB} =& A \cap B - C\\
			S_{AC} =& A \cap C - C\\
			S_{ABC} =& A \cap B \cap C 
		\end{align*}
		and the other terms are defined analogously.
		
		Now in \eqref{eq:symm:diff:associative}
		\begin{align*}
			A \setminus B =& \big[\big(S_A \cancel{\cup S_{AB}} \cup S_{AC} \cup \bcancel{S_{ABC}}\big) - \big(S_B \cup S_{BC} \cancel{\cup S_{BA}} \bcancel{\cup S_{BCA}}\big)\big]\\
			=& \big[\big(S_A  \cup S_{AC}\big) - \big(S_B \cup S_{BC} \big)\big]\\
			B \setminus A =& \big[\big(S_B  \cup S_{BC}\big) - \big(S_A \cup S_{AC} \big)\big]\\
			A \triangle B =& \bigg[\big[\big(S_A  \cup S_{AC}\big) - \big(S_B \cup S_{BC} \big)\big] \bigcup \big[\big(S_B  \cup S_{BC}\big) - \big(S_A \cup S_{AC} \big)\big]\bigg]
		\end{align*}
		
		\begin{align*}
					A \triangle B 32=& \big(A \cap \stcomp{B}\big) \bigcup \big(B \cap \stcomp{A}\big)
		\end{align*}
	\end{proof}
	\begin{thm}[$	A \triangle C =\big(A\triangle B\big) \triangle \big(B\triangle C\big)$]
		\begin{align*}
			\forall& A,B,C\\
			A \triangle C =& \big(A\triangle B\big) \triangle \big(B\triangle C\big) \numberthis \label{eq:symm:diff:3}
		\end{align*}
	\end{thm}
	\begin{proof}
		We begin by applying \eqref{eq:symm:diff:associative} to the RHS of \eqref{eq:symm:diff:3}.
		\begin{align*}
			\big(A\triangle B\big) \triangle \big(B\triangle C\big) =& A \triangle \big[B \triangle \big(B\triangle C\big)\big] \text{, then apply \eqref{eq:symm:diff:associative} again:}\\
			=& A \triangle \underbrace{\big[\big(\underbrace{B \triangle B}_\text{$=\emptyset$}\big)\triangle C\big]}_\text{$=C$}\\
			=&A \triangle C
		\end{align*}
	\end{proof}
	\begin{cor}[$A \triangle C \subseteq \big(A\triangle B\big) \bigcup \big(B\triangle C\big)$]
			\begin{align*}
			\forall& A,B,C\\
			A \triangle C \subseteq& \big(A\triangle B\big) \bigcup \big(B\triangle C\big)
		\end{align*}
	\end{cor}
\end{appendices}

\printglossaries

% bibliography goes here

\bibliographystyle{unsrt}
\addcontentsline{toc}{section}{Bibliography}
\bibliography{ActiveInference}

\end{document}
